%!TEX program = xelatex
\documentclass[]{article}
\usepackage{amsmath}
\usepackage[UTF8]{ctex}

% bold style for matrix notation
\newcommand{\matr}[1]{\mathbf{#1}}  % undergraduate algebra version
%\newcommand{\matr}[1]{#1}          % pure math version
%\newcommand{\matr}[1]{\bm{#1}}     % ISO complying version

%opening
\title{矩阵}
\author{}

\begin{document}

\maketitle

\begin{abstract}
常用线性代数的引理和定理,写着玩
\end{abstract}

\section{实对称矩阵的特征值是实数}
\begin{equation}\label{key}
\begin{aligned}
Ax &= \lambda x \\
x^H A^H &= \bar{\lambda} x^H \\
x^H A^H x &= \bar{\lambda} x^H x \\
A^H &= A \\
x^H A x &= \bar{\lambda} x^H x \\
\lambda x^H x &= \bar{\lambda} x^H x \\
\lambda &= \bar{\lambda}
\end{aligned}
\end{equation}

\section{n阶方阵一定有n个特征根(重跟按重数计算)}
设A是一个n阶方阵,它的特征值多项式$ \left| A - \lambda I \right| $是一个关于$ \lambda $ 的多项式。根据代数基本定理,它可以唯一的分解成一次因式的乘积。所以$ \left| A - \lambda I \right| $一定有n个复数跟。

\section{n阶实对称矩阵一定有n个实特征跟(重跟按重数计算)}
由以上两条结论易证。

\section{实对称矩阵,不同特征值的特征向量正交}
\begin{equation}\label{key}
\begin{aligned}
A x_1 &= \lambda_1 x_1\\
A x_2 &= \lambda_2 x_2\\
x_1^H A^H &= \lambda_1 x_1^H \\
x_1^H A^H x_2 &= \lambda_1 x_1^H x_2 \\
\lambda_2 x_1^H x_2 &= \lambda_1 x_1^H x_2 \\
(\lambda_1 - \lambda_2) x_1^H x_2 &= 0 \\
\lambda_1 &\neq \lambda_2 \\
x_1^H x_2 &= 0
\end{aligned}
\end{equation}

\section{设A为n阶对称矩阵,则必有正交矩阵P,使得$ P^{-1}AP = P^TAP = B $, 其中B是以A的特征值为对角线元素的对角矩阵}
也即对称矩阵与对角矩阵相似。不会证。

\section{实对称矩阵是半正定矩阵的充分必要条件是它的所有特征值都非负}
由以上结论不难证明。\\
正定矩阵定义:一个$ n \times n $的实对称矩阵A是正定的,当且仅当对于所有的非零实系数向量x,都有$ x^T A x > 0 $.



\end{document}
