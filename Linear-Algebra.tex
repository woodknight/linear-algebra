%!TEX program = xelatex
\documentclass[]{article}
\usepackage{amsmath}
\usepackage[UTF8]{ctex}

% bold style for matrix notation
\newcommand{\matr}[1]{\mathbf{#1}}  % undergraduate algebra version
%\newcommand{\matr}[1]{#1}          % pure math version
%\newcommand{\matr}[1]{\bm{#1}}     % ISO complying version

%opening
\title{矩阵}
\author{}

\begin{document}

\maketitle

\begin{abstract}
常用线性代数的引理和定理,写着玩
\end{abstract}

\section{实对称矩阵的特征值是实数}
\begin{equation}\label{key}
\begin{aligned}
Ax &= \lambda x \\
x^H A^H &= \bar{\lambda} x^H \\
x^H A^H x &= \bar{\lambda} x^H x \\
A^H &= A \\
x^H A x &= \bar{\lambda} x^H x \\
\lambda x^H x &= \bar{\lambda} x^H x \\
\lambda &= \bar{\lambda}
\end{aligned}
\end{equation}

\section{n阶方阵一定有n个特征根(重跟按重数计算)}
设A是一个n阶方阵,它的特征值多项式$ \left| A - \lambda I \right| $是一个关于$ \lambda $ 的多项式。根据代数基本定理,它可以唯一的分解成一次因式的乘积。所以$ \left| A - \lambda I \right| $一定有n个复数跟。

\section{n阶实对称矩阵一定有n个实特征跟(重跟按重数计算)}
由以上两条结论易证。

\section{实对称矩阵,不同特征值的特征向量正交}
\begin{equation}\label{key}
\begin{aligned}
A x_1 &= \lambda_1 x_1\\
A x_2 &= \lambda_2 x_2\\
x_1^H A^H &= \lambda_1 x_1^H \\
x_1^H A^H x_2 &= \lambda_1 x_1^H x_2 \\
\lambda_2 x_1^H x_2 &= \lambda_1 x_1^H x_2 \\
(\lambda_1 - \lambda_2) x_1^H x_2 &= 0 \\
\lambda_1 &\neq \lambda_2 \\
x_1^H x_2 &= 0
\end{aligned}
\end{equation}

\section{设A为n阶对称矩阵,则必有正交矩阵P,使得$ P^{-1}AP = P^TAP = B $, 其中B是以A的特征值为对角线元素的对角矩阵}
也即对称矩阵与对角矩阵相似。不会证。为什么A一定会有n个线性无关的特征向量?

\section{实对称矩阵是半正定矩阵的充分必要条件是它的所有特征值都非负}
由以上结论不难证明。\\
正定矩阵定义:一个$ n \times n $的实对称矩阵A是正定的,当且仅当对于所有的非零实系数向量x,都有$ x^T A x > 0 $.

\section{n阶方阵}
\subsection{可对角化的条件}
n阶方阵可对角化的充要条件是有n个线性无关的特征向量。
如果n阶方阵有n个不同的特征值,则n可对角化,反之不一定对。考虑矩阵
\begin{equation}
M = 
\begin{bmatrix}
-1 & 3 & -1 \\
-3 & 5 & -1 \\
-3 & 3 & 1
\end{bmatrix}
\end{equation}
M的特征值为(1,2,2), 重根2有两个线性无关的特征向量,(1,1,0)和(-1/3,0,1). 所以M有3个线性无关的的特征向量,可以对角化。所谓的代数重数和几何重数。

\subsection{对角化的方法 - 特征值分解}
假设n阶矩阵$ A $有特征根$\lambda_1,  \lambda_2, \cdots, \lambda_n$,以及n个线性无关的特征向量 $ q_1, q_2, \cdots, q_n $(列向量),易证明
\[ 
Q^{-1}AQ = \Lambda
 \]
其中
\[ 
\Lambda =
\begin{bmatrix}
\lambda_1 & & & \\
& \lambda_2 & & \\
& & \ddots & \\
& & & \lambda_n
\end{bmatrix}
 \]
\[ 
Q = [q_1, q_2, \cdots, q_n]
 \]
证明如下
\[ 
\begin{aligned}
AQ &= A[q_1, q_2, \cdots, q_n] \\
   &= [Aq_1, Aq_2, \cdots, Aq_n] \\
   &= [\lambda_1 q_1, \lambda_2 q_2, \cdots, \lambda_n q_n] \\
   &= 
   \begin{bmatrix}
   \lambda_1 & & & \\
   & \lambda_2 & & \\
   & & \ddots & \\
   & & & \lambda_n
   \end{bmatrix}   
   [q_1, q_2, \cdots, q_n] \\
   &= \Lambda Q
\end{aligned}
 \]
所以有
\[ 
A = Q\Lambda Q^{-1}
 \]
\textbf{特征值分解可以用来求矩阵的逆。}可以看出
\[ 
A^{-1} = Q\Lambda^{-1}Q^{-1}
 \]
因为$ \Lambda $是对角矩阵,所以有
\[ 
\Lambda^{-1} =
\begin{bmatrix}
\lambda_1^{-1} & & & \\
& \lambda_2^{-1} & & \\
& & \ddots & \\
& & & \lambda_n^{-1}
\end{bmatrix}
\]
\textbf{当矩阵是对称方阵时}
\[ 
\begin{aligned}
A &= A^T \\
Q\Lambda Q^{-1} &= (Q\Lambda Q^{-1})^T \\
                &= (Q^{-1})^T\Lambda^T Q^T \\
                &= (Q^{-1})^T\Lambda Q^T
\end{aligned}
 \]
于是有
\[ 
Q^{-1} = Q^T
 \]

\end{document}