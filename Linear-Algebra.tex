%!TEX program = xelatex
\documentclass[]{article}
\usepackage{amsmath}
\usepackage[UTF8]{ctex}

% bold style for matrix notation
\newcommand{\matr}[1]{\mathbf{#1}}  % undergraduate algebra version
%\newcommand{\matr}[1]{#1}          % pure math version
%\newcommand{\matr}[1]{\bm{#1}}     % ISO complying version

%opening
\title{矩阵}
\author{}

\begin{document}

\maketitle

\begin{abstract}
常用线性代数的引理和定理,写着玩
\end{abstract}

\section{实对称矩阵的特征值是实数}
\begin{equation}\label{key}
\begin{aligned}
Ax &= \lambda x \\
x^H A^H &= \bar{\lambda} x^H \\
x^H A^H x &= \bar{\lambda} x^H x \\
A^H &= A \\
x^H A x &= \bar{\lambda} x^H x \\
\lambda x^H x &= \bar{\lambda} x^H x \\
\lambda &= \bar{\lambda}
\end{aligned}
\end{equation}

\section{n阶方阵一定有n个特征根(重跟按重数计算)}
设A是一个n阶方阵,它的特征值多项式$ \left| A - \lambda I \right| $是一个关于$ \lambda $ 的多项式。根据代数基本定理,它可以唯一的分解成一次因式的乘积。所以$ \left| A - \lambda I \right| $一定有n个复数跟。

\section{n阶实对称矩阵一定有n个实特征跟(重跟按重数计算)}
由以上两条结论易证。

\section{实对称矩阵,不同特征值的特征向量正交}
\begin{equation}\label{key}
\begin{aligned}
A x_1 &= \lambda_1 x_1\\
A x_2 &= \lambda_2 x_2\\
x_1^H A^H &= \lambda_1 x_1^H \\
x_1^H A^H x_2 &= \lambda_1 x_1^H x_2 \\
\lambda_2 x_1^H x_2 &= \lambda_1 x_1^H x_2 \\
(\lambda_1 - \lambda_2) x_1^H x_2 &= 0 \\
\lambda_1 &\neq \lambda_2 \\
x_1^H x_2 &= 0
\end{aligned}
\end{equation}

\section{设A为n阶对称矩阵,则必有正交矩阵Q,使得$ Q^{-1}AQ = Q^TAQ = \Lambda $, 其中$ \Lambda $是以A的特征值为对角线元素的对角矩阵}
也即对称矩阵与对角矩阵相似。不会证。为什么A一定会有n个线性无关的特征向量?

\section{实对称矩阵是半正定矩阵的充分必要条件是它的所有特征值都非负}
由以上结论不难证明。\\
正定矩阵定义:一个$ n \times n $的实对称矩阵A是正定的,当且仅当对于所有的非零实系数向量x,都有$ x^T A x > 0 $.

\section{对称矩阵}
\subsection{任意一个n阶方阵都可以表示成一个对称矩阵和一个反对称矩阵之和}
\[ A = \frac{1}{2}((A+A^T)+(A-A^T)) \]

\section{满秩}
\subsection{n阶方阵A是满秩的当且仅当A可以表示为n阶初等矩阵的乘积}
必要性:一个满秩的n阶方阵总可以经过若干次初等行变换和初等列变化化为标准型,故存在初等矩阵,使
\[
\begin{aligned}
A &= P_1P_2\cdots P_s E Q_1 Q_2 \cdots Q_l \\
  &= P_1P_2\cdots P_s Q_1 Q_2 \cdots Q_l
\end{aligned}
\]

充分性:如果A可以表示为n阶初等矩阵的乘积,则
\[
\begin{aligned}
	A &= P_1P_2\cdots P_s \\
	  &= P_1P_2\cdots P_s E
\end{aligned}
\]
其中E为n阶单位矩阵。因为初等变换不改变矩阵的秩,所以$ r(A) = r(E) = n $, 即A是满秩的。
\section{对角矩阵}
用一个m阶对角矩阵$ D_m $左乘一个$ m\times n $矩阵$ A = (a_{ij}) $, 所得结果相当于分别用$ d_1, d_2, \cdots, d_m $乘以$ A $的第$ 1,2,\cdots,m $行。

用一个n阶对角矩阵右
乘$ A $,所得结果相当于分别用$ d_1, d_2, \cdots, d_n $乘以$ A $的第$ 1,2,\cdots,n $列。

\section{矩阵乘积的秩}
\subsection{给定矩阵$ A_{m\times n}, B_{n\times s} $, 那么有$r(AB)\leq min\{r(A),r(B)\}$}
设$ AB=C $, 则C的每个行向量都是B的行向量的线性组合,所以$ r(C)\leq r(B) $. 同理C的每个列向量都是A的列向量的线性组合,所以$ r(C)\leq r(A) $.
\subsection{设A是一个$ m\times n $阶矩阵,P是一个m阶满秩方阵,Q是一个n阶满秩方阵,则$ r(PA)=r(AQ)=r(A)$, 即左乘或右乘满秩方阵后,矩阵的秩不变.}
由上面的命题,不难证明。也可证明如下。
P是满秩方阵,所以P可以表示成初等矩阵的乘积,而初等变化不改变矩阵的秩,所以
\[ 
r(PA) = r(P_1P_2\cdots P_sA) = r(A)
 \]
同理可证\[ r(AQ) = r(A) \]

\section{矩阵乘积的行列式}
\subsection{设$ P $是$ n $阶初等矩阵,A是任意n阶方阵,则$ det(PA)=det(P)det(A) $}
\subsection{对任意n阶方阵$ A $,$ B $,有$ det(AB)=det(A)det(B) $}

\section{可逆矩阵}
\subsection{一个n阶方阵A可逆的充分必要条件是,A是满秩的。}
证:

(i)必要性:当A可逆时,存在B,使得$ AB=E $, 由于
\[
\begin{aligned}
r(AB) &\leq min(r(A), r(B)) \\
r(AB) &= r(E) = n 
\end{aligned}
\]
所以$ r(A) = r(B) = n $.

(ii)充分性:若A是满秩的,则A可以表示成初等矩阵的乘积,初等矩阵是可逆的,它们的乘积也是可逆的,故A是可逆的。

\section{相似变换}
当存在一个可逆矩阵P,使得
\[ 
B = P^{-1}AP
 \]
则称A和B为相似矩阵,P称为相似变换矩阵。

相似矩阵A和B有许多相同的性质。
\subsection{秩相等}
P为可逆矩阵,所以P是满秩的,矩阵乘以满秩矩阵,不改秩数。
\subsection{行列式值相等}
\subsection{迹相等}
\subsection{相同的特征多项式}
\[ 
\begin{aligned}
det(B-\lambda I) &= det(P^{-1}AP - \lambda I) \\
				 &= det(P^{-1}AP - \lambda P^{-1}IP) \\
				 &=	det(P^{-1})det(AP - \lambda I)det(P) \\
				 &= det(P^{-1}P)det(AP - \lambda I) \\
				 &=	det(A - \lambda I)
\end{aligned}
 \]
A和B有相同的特征多项式,所以有相同的根,即特征值。
 
\subsection{特征值相同,特征向量一般不同}
\[ 
\begin{aligned}
Ax &= \lambda x \\
PBP^{-1}x &= \lambda x \\
B(P^{-1}x) &= \lambda (P^{-1}x)
\end{aligned}
 \]
$ A $的特征向量$ x $对应于$ B $的特征向量$ P^{-1}x $.

因为A与B有许多相似的性质,因此在给定了A以后,如果能找到一个与之相似而形式又足够简单的矩阵B, 那么对A的研究就可以转化为对更为简单的B的研究,使计算简化许多。比如说如果A与一个对角矩阵相似,就说A是可对角化的。

\section{n阶方阵}
\subsection{可对角化的定义}
如果n阶方阵A与一个对角矩阵$\Lambda$相似,就说A是可对角化的。
\[ P^{-1}EP = A  \]

注意,所有的矩阵都可以通过一系列的初等行变换和列变化,化为标准型,不要和可对角化混淆。
\[ P_1 P_2\cdots P_s A Q_1 Q_2 \cdots Q_l = D \]

\subsection{可对角化的条件}
n阶方阵可对角化的充要条件是有n个线性无关的特征向量。证明见下文。
如果n阶方阵有n个不同的特征值,则n可对角化,反之不一定对。考虑矩阵
\begin{equation}
A = 
\begin{bmatrix}
-1 & 3 & -1 \\
-3 & 5 & -1 \\
-3 & 3 & 1
\end{bmatrix}
\end{equation}
A的特征值为(1,2,2), 重根2有两个线性无关的特征向量,(1,1,0)和(-1/3,0,1). 所以A有3个线性无关的的特征向量,可以对角化。所谓的代数重数和几何重数。

\subsection{对角化的方法 - 特征值分解}
假设n阶矩阵$ A $有特征根$\lambda_1,  \lambda_2, \cdots, \lambda_n$,以及n个线性无关的特征向量 $ q_1, q_2, \cdots, q_n $(列向量),易证明
\[ 
Q^{-1}AQ = \Lambda
 \]
其中
\[ 
\Lambda =
\begin{bmatrix}
\lambda_1 & & & \\
& \lambda_2 & & \\
& & \ddots & \\
& & & \lambda_n
\end{bmatrix}
 \]
\[ 
Q = [q_1, q_2, \cdots, q_n]
 \]
证明如下
\[ 
\begin{aligned}
AQ &= A[q_1, q_2, \cdots, q_n] \\
   &= [Aq_1, Aq_2, \cdots, Aq_n] \\
   &= [\lambda_1 q_1, \lambda_2 q_2, \cdots, \lambda_n q_n] \\
   &=  
   [q_1, q_2, \cdots, q_n] 
   \begin{bmatrix}
   \lambda_1 & & & \\
   & \lambda_2 & & \\
   & & \ddots & \\
   & & & \lambda_n
   \end{bmatrix}   \\
   &= Q\Lambda
\end{aligned}
 \]
所以有
\[ 
A = Q\Lambda Q^{-1}
 \]
\textbf{特征值分解可以用来求矩阵的逆。}可以看出
\[ 
A^{-1} = Q\Lambda^{-1}Q^{-1}
 \]
因为$ \Lambda $是对角矩阵,所以有
\[ 
\Lambda^{-1} =
\begin{bmatrix}
\lambda_1^{-1} & & & \\
& \lambda_2^{-1} & & \\
& & \ddots & \\
& & & \lambda_n^{-1}
\end{bmatrix}
\]
\textbf{当矩阵是对称方阵时}
\[ 
\begin{aligned}
A &= A^T \\
Q\Lambda Q^{-1} &= (Q\Lambda Q^{-1})^T \\
                &= (Q^{-1})^T\Lambda^T Q^T \\
                &= (Q^{-1})^T\Lambda Q^T
\end{aligned}
 \]
于是有
\[ 
Q^{-1} = Q^T
 \]
也就是说,Q是正交矩阵。


\section{矩阵的奇异值分解SVD}
假设A是一个$ m \times n$的矩阵,那么定义矩阵A的SVD为:
\[ 
A = U\Sigma V^T
 \]
其中U是$ m\times m $的矩阵,V是$ n\times n $的矩阵。U和V都是幺正矩阵(酉矩阵, unitary matrix),即$ U^T U = I $, $ V^T V = I $。$ \Sigma $是$ m \times n$的矩阵,除了对角线以外的元素都是0,对角线上的元素称为奇异值。

$ A^T A $是$ n\times n $的一个对称方阵,又上文可知,可以对其做特征值分解,得到n个正交的特征向量,所有的特征向量可以张成一个$ n\times n $的幺正矩阵$V$.

用类似的方法,对$ AA^T $做特征值分解,可以得到$ m\times m $的幺正矩阵$U$.

这里得到的$ U $和$ V $就是上面SVD公式里面的$ U $和$ V $. 证明如下:
\[ 
\begin{aligned}
A     &= U\Sigma V^T \\
A^T   &= V\Sigma U^T \\
AA^T  &= U\Sigma^2 U^T
\end{aligned}
 \]
这个链接写的很好:
http://www.cnblogs.com/pinard/p/6251584.html



\end{document}